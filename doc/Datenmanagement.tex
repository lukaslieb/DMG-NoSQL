\section{Datenmanagement}
Zwei Aktoren wirken auf die Datenbank ein. Einerseits der Benutzer, welcher die
Informationen abfragen kann. Andererseits der DB-Administrator, der die
Daten abfragen, aber auch verändern und neue hinzufügen kann. Diese
Anwendungsfälle sind in der Abbildung \ref{fig:usecase} abgebildet.

\begin{figure}[h]
  \centering
     \includegraphics[width=1\textwidth]{./pictures/UseCase.png}
  \caption{Visualisierung des Anwendungsfalls}
  \label{fig:usecase}
\end{figure}
\noindent
Die Daten, welche in der MongoDB zum Einsatz kamen, wurden aus der Uni-DB, die für einige Testataufgaben im Modul DMG bereits verwendet wurde, migriert. Um diese Migration durchzuführen ist ein Java Programm geschrieben worden, welches in Kapitel \ref{kap:Datenbanksprachen} näher erläutert wird. Der Strukturierte Aufbau der Datenbank wird in Kapitel \ref{kap:ERDiagramm} beschrieben.

\newpage
\noindent
Der Benutzer hat die Möglichkeit über ein GUI Daten auf der Datenbank abzufragen:
\begin{figure}[h]
	\centering
	\includegraphics[width=0.5\textwidth]{./pictures/UniDBView.png}
	\caption{Gui für den Benutzer}
	\label{fig:GUI1}
\end{figure}
\noindent
Über dieses GUI sind alle Professoren ersichtlich welche in der Datenbank eingetragen sind.
Mithilfe des SearchStudents Knopf kann der Benutzer abfragen, wie viele Studenten in der Vorlesung eines Professors sitzen, und wie viele SWS Punkte er unterrichtet.