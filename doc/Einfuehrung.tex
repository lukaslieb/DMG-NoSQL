\section{Einführung}

\subsection{Aufgabenstellung}
Ziel dieser Arbeit ist es, die Uni-DB, welche auf einem SQL Server liegt, in eine NoSQL-Datenbank zu implementieren. Dazu wird das ER-Schema der Uni-DB so umgezeichnet, dass es in der gewählten NoSQL-Datenbank abgebildet werden kann. Dieses Schema wird dann in der entsprechend gewählten Datenbank umgesetzt. Dazu migriert man die Daten aus der SQL- in die NoSQL-Datenbank.
Zusätzlich wird dem Benutzer eine Möglichkeit geboten, mit Hilfe eines GUI folgende SQL-Querry auf der MongoDB abzufragen:


\lstinputlisting{sourcecode/SQLQuerry.cpp}
%\begin{lstlisting}
%select ProfessorName, AnzahlStudenten, SummeSWS 
%from ( 
%select p.Name as ProfessorName, count(s.MatrNr) 
%as AnzahlStudenten 
%from Professoren p 
%join Vorlesungen v on v.gelesenVon = p.PersNr
%join hoeren h on h.VorlNr = v.VorlNr 
%join Studenten s on s.MatrNr = h.MatrNr 
%group by p.Name 
%) A 
%join 
%( 
%select p.Name as ProfessorName,  sum(SWS) 
%as summeSWS 
%from Professoren p 
%join Vorlesungen v on v.gelesenVon = p.PersNr 
%group by p.Name 
%) B using(ProfessorName)
%\end{lstlisting}

\newpage
\subsection{Datenbankauswahl}
Bei der Auswahl der Datenbank haben wir uns für die MongoDB entschieden. Ausschlaggebend waren dabei folgende Gründe:
\begin{itemize}
  \item Geeignet um unser Problem zu lösen.
  \item Wird in der Praxis eingesetzt
  \item Bekannt
  \item Gute Dokumentation
  \item Kostenlos
  \item Unterstützung durch Forenuser
  \item Erste Erfahrung vorhanden
\end{itemize}

Die MongoDB ist eine Datenbank, welche sich grundlegend im CAP Theorem in der Spalte CP aufhält. Dies bedeutet, dass die Konsitenz (Consistency) gewährleistet ist. Dies wird realisiert, indem in einem System ein Knoten als primäres Mitglied fungiert und alle Anfragen über diesen Knoten abgearbeitet werden. Zusätzlich wird die Partition-Tolerance gewährleistet. 
Gemäss Betreiber wird bei einem Ausfall des Primärknotens automatisch ein Sekundärknoten zum neuen Primärknoten definiert und somit eine einwandfreie Funktion gewährleistet. Ebenfalls soll es möglich sein, auf Kosten der Konsistenz (Consistency) die Verfügbarkeit (Availability) zu erhöhen, indem man in den Einstellungen die Konfiguration vornimmt, dass Daten nicht bloss über den Primär-Knoten, sondern auch über Sekundärknoten bezogen werden können.



