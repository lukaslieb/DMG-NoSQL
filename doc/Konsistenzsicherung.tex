 \section{Konsistenzsicherung}
 Bei relationalen Datenbanken bietet die Datenbank Mechanismen zur
 Konsistenzsicherung an. Dies wird durch verschiedene Eigenschaften im RDBMS
 erreicht. Jeder Tabelle liegt ein Schema zu Grunde. In diesem wird definiert,
 wie die Daten strukturiert sein müssen. Zum Beispiel wird festgelegt, welches
 der Primärschlüssel ist, ob Felder ``null'' sein dürfen, oder ob Tupel gelöscht
 werden dürfen, wenn noch Referenzen darauf zeigen.
 MongoDB ist Schema frei. Das Bedeutet, dass sich je zwei Dokumente in einer
Sammlung  komplett in ihrer Struktur voneinander unterscheiden können.
 Des Weitern garantiert MongoDB keine integrität der Referenzen. Dieser Umstand
 wird klar, wenn man vergleicht, welchen Regeln die relationale Datenbank und
 die MognoDB unterliegen.
 RDBMS unterliegen den ACID Regeln. 
 \begin{itemize}
   \item Atomar: Die gesamte Transaktion wird ausgeführt oder die Transaktion
   wird rückgängig gemacht.
   \item Consistent: Nach jeder Transaktion muss die Datenbank widerspruchsfrei
   sein.
   \item Isolatet: Bei einer Transaktion dürfen keine Seiteneffekte
   auftreten. Dies garantiert, dass ein Mehrbenutzerbetrieb möglich ist.
   \item Dauerhaft: Die Daten werden sicher gespeichert, auch bei
   Systemabstürzen. Bei einem Systemabsturz wird ein recovery durchgeführt, so 
   dass danach die Datenbank wieder in einem konsistenten Zustand ist.
 \end{itemize}
 Die in unserem Projekt eingesetzte MongoDB unterliegt den BASE Regeln.
 \begin{itemize}
   \item Basically Available: Die Datenbank sollte meistens laufen.
   \item Eventually Consistent: Die Konsistenz der Daten wird nicht unmittelbar
   nach der Operation gewährleistet. Sie kann verzögert eintreten.
\end{itemize}
Da MongoDB keine Transaktionen(eine Folge von Operationen, die Atomar
ausgeführt werden)) unterstützt, muss diese Funktionalität auf der
Anwendungsebene implementiert werden. Dies ist beim Einsatz eines RDBMS nicht
notwendig, da dieses Transaktionen unterstützt. In unserem Fall werden keine
Transaktionen verwendet, da nur lesend auf die Daten zugegriffen werden.
Da nur gelesen wird, und keine Daten geändert oder hinzugefügt/entfernt werden,
kann die Datenbank durch Operationen nicht in einen inkonsistenten Zustand
überführt werden. Deswegen kann auch parallel auf die Daten zugegriffen
werden, ohne die Konsistenz zu verlieren. Sollen zu einem späteren Zeitpunkt zur
Laufzeit der Anwendung Daten geändert oder hinzugefügt/entfernt werden,
 so muss eine Konsistenzsicherung auf der Anwendungsebene  implementiert werden.


 

		